\RequirePackage[l2tabu, orthodox]{nag} % Check if some command or packages are obsolete
\RequirePackage{fix-cm} % Recent fixes for LaTeX2e

\documentclass[dvipsnames,svgnames,doctor,final,11pt]{new-iscs-thesis}
% Choose between 'senior' for bachellor thesis, 'master' for master thesis and 'doctor' for doctoral thesis.
% Choose between 'draft' for the work-in-progress pdf generations presentation and 'final' for the final submition
% Choose 'interim' for the interim presentation
% Choose 'english' for

%----------------------------------------------------------------------------------------
%	PACKAGES
%----------------------------------------------------------------------------------------
% For direct Japanese input. Compilation with XeLaTex required
%\usepackage{xeCJK} 
%\setCJKmainfont{Hiragino Mincho Pro} % for \rmfamily
%\setCJKsansfont{Hiragino Kaku Gothic Pro} % for \sffamily
%\setCJKmainfont{IPAPMincho} % for \rmfamily
%\setCJKsansfont{IPAPGothic} % for \sffamily
%\setCJKfamilyfont{songti}{SimSun} % used as \CJKfamily{songti}

% For direct Japanese input. Using luatex.
\usepackage{luatexja-fontspec}

%\setmainjfont{YuMincho Medium} % \mcfamily
%\setsansjfont{YuGothic Medium} % \gtfamily
%\fontencoding{JY3}
%\fontseries{yu-osx} 
%\fontshape{}

%\setmainjfont{MS Mincho} % \mcfamily
%\setsansjfont{MS Gothic} % \gtfamily

% AMS packages
\usepackage{amsmath} % AMS math 
\usepackage{amssymb} % AMS symbols
\usepackage{amsfonts} % AMS fonts
\usepackage{amsthm} % AMS theorem and alike

% Other mathematical fonts
\usepackage{mathrsfs}

% etoolbox package: boolean, CQFD, ...
\usepackage{etoolbox}

% Algorithm package
\usepackage[lined,boxed,linesnumbered]{algorithm2e}

% TiKz
\usepackage{tikz}
\usepackage{pgfplots}
\pgfplotsset{compat=1.9}
\usetikzlibrary{shapes.misc}
\usepackage{makecell}

% Recent fixes to LaTeX2e
\usepackage{fixltx2e}

% Pdf metadata
\usepackage{datetime}
\pdfinfo{
   /Author 	(TODAI Taro)
   /Keywords 	(topic 1;  topic 2; topic 3)
   /ModDate	(D:\pdfdate)
   /Title 	(A Lua-LaTeX thesis template for IS/CS at U-Tokyo)
}

% Improve spacement between words (to add after language package, here luatexja)
\usepackage{microtype}

% Improve the tables. Use \toprule, \midrule and \bottomrule instead of \hrule
\usepackage{booktabs}

% Additional itemize environment: \compactitem \compactenum
\usepackage{paralist}

% To force a float to appear in the section (or above on the page starting the section
\usepackage[section, above]{placeins}

% To have nice fraction, specially inline: \nicefrac[]{1}{2} for 1/2
\usepackage[]{nicefrac}

% To superscript number^nth automatically (do not requiere to be in math mode)
\usepackage{nth}

% Appendix
\usepackage[toc,page]{appendix}

% Subfloats with \subfigure and \subtable
\usepackage{caption}
\usepackage{subcaption}

%----------------------------------------------------------------------------------------
%	NEW COMMANDS
%----------------------------------------------------------------------------------------

% Sets
\newcommand{\C}{\mathscr{C}}
\newcommand{\F}{\mathscr{F}}
\newcommand{\R}{\mathbb{R}}

% Variables
\newcommand{\funF}{$\mathcal{F}$}

% Set unions
\makeatletter
\def\moverlay{\mathpalette\mov@rlay}
\def\mov@rlay#1#2{\leavevmode\vtop{%
    \baselineskip\z@skip \lineskiplimit-\maxdimen
    \ialign{\hfil$\m@th#1##$\hfil\cr#2\crcr}}}
\newcommand{\charfusion}[3][\mathord]{
  #1{\ifx#1\mathop\vphantom{#2}\fi
    \mathpalette\mov@rlay{#2\cr#3}
  }
  \ifx#1\mathop\expandafter\displaylimits\fi}
\makeatother
\newcommand{\cupdot}{\charfusion[\mathbin]{\cup}{\cdot}}
\newcommand{\bigcupdot}{\charfusion[\mathop]{\bigcup}{\cdot}}

% new theorem and alike
\theoremstyle{plain}
\newtheorem{thm}{Theorem}
\newtheorem{lem}{Lemma}
\newtheorem{prop}{Proposition}
\newtheorem{cor}{Corollary}

\theoremstyle{definition}
\newtheorem{defn}{Definition}
\newtheorem{conj}{Conjecture}
\newtheorem{exmp}{Example}

\theoremstyle{remark}
\newtheorem*{rem}{Remark}
\newtheorem*{note}{Note}
\newtheorem{case}{Case}

%----------------------------------------------------------------------------------------
%	TIKZ STYLES
%----------------------------------------------------------------------------------------
\tikzset{RootStyle/.style = % For source and sink
{
  shape          = circle,
  draw           = red!50!black!50,
  thick,
  top color      = white,
  bottom color   = red!50!black!20,
  text           = black,
  inner sep      = .2pt,
  outer sep      = 0pt,
  minimum size   = 2.5 mm
}}
\tikzset{VertexStyle/.style = % For normal nodes
{
  shape          = circle,
  draw           = black!50,
  top color      = white,
  bottom color   = black!20,
  text           = black,
  inner sep      = .2pt,
  outer sep      = 0pt,
  minimum size   = 1.75 mm
}}
\tikzset{EdgeStyle/.style  = 
{
}}
\tikzset{LabelStyle/.style =   
{%draw, thin,
  %fill           = blue!10,
  text           = black,
  inner sep      = .2pt,
  outer sep      = 1pt,
  font           = \small,
  minimum size   = 2.15 mm
}}
\tikzset{LegendStyle/.style =
{
  shape          = rectangle,
  draw           = black!90,
  top color      = white,
  bottom color   = black!20,
  text           = black,
  inner sep      = 3pt,
  outer sep      = 0pt,
  minimum size   = 15 mm
}}
%----------------------------------------------------------------------------------------
%	OTHER SETUP
%----------------------------------------------------------------------------------------
\graphicspath{{images/}}

% Math Operators
\DeclareMathOperator*{\argmin}{arg\,min}
\DeclareMathOperator*{\argmax}{arg\,max}

% To adapt the placement of floats
% \renewcommand{\textfraction}{0.05}
% \renewcommand{\topfraction}{0.8}
% \renewcommand{\bottomfraction}{0.8}
% \renewcommand{\floatpagefraction}{0.75}

%----------------------------------------------------------------------------------------
%----------------------------------------------------------------------------------------
%	DYNAMIC PREAMBLE
%----------------------------------------------------------------------------------------
%----------------------------------------------------------------------------------------
\newbool{static_preamble}
\setbool{static_preamble}{false}
\ifbool{static_preamble}{%true
  \csname endofdump\endcsname
}{}%false

% ToDoNotes
% Documentation at:
% http://www.tex.ac.uk/ctan/macros/latex/contrib/todonotes/todonotes.pdf
\newbool{use_todo}
\setbool{use_todo}{true}
\ifbool{use_todo}{%true
  \usepackage[colorinlistoftodos, shadow]{todonotes}
  \setlength{\marginparwidth}{2.8cm}
}{%false
\usepackage[disable]{todonotes}
}

% Syntax only
\newbool{syntax_only}
\setbool{syntax_only}{false}
\ifbool{syntax_only}{%true
  \usepackage{syntonly}
  \syntaxonly
}{}%false

% Appendix boolean
\newbool{print_appendix}
\setbool{print_appendix}{true}

%----------------------------------------------------------------------------------------
%	TIKZ EXTERNAL
%----------------------------------------------------------------------------------------
\newbool{print_fig_external}
\setbool{print_fig_external}{false}
\ifbool{print_fig_external}{%true
  \usepgfplotslibrary{external}
  \tikzexternalize[prefix=figures/]
  \tikzset{external/system call={lualatex -shell-escape -halt-on-error -interaction=batchmode -jobname "\image" "\texsource"}}
}{}%false

% todo commands
\ifbool{print_fig_external}{%true
\newcommand{\toreader}[2]{\tikzexternaldisable\todo[inline,color=blue!30,caption={#1}]{ #2 }\tikzexternalenable}
\newcommand{\critical}[2]{\tikzexternaldisable\todo[inline,color=red!30,caption={#1}]{ #2 }\tikzexternalenable}
\newcommand{\related}[2]{\tikzexternaldisable\todo[inline,color=green!30,caption={#1}]{ #2 }\tikzexternalenable}
\newcommand{\normaltodo}[2]{\tikzexternaldisable\todo[inline,color=orange!30,caption={#1}]{ #2 }\tikzexternalenable}
\newcommand{\otoreader}[2]{\tikzexternaldisable\todo[color=blue!30,caption={#1}]{ #2 }\tikzexternalenable}
\newcommand{\ocritical}[2]{\tikzexternaldisable\todo[color=red!30,caption={#1}]{ #2 }\tikzexternalenable}
\newcommand{\orelated}[2]{\tikzexternaldisable\todo[color=green!30,caption={#1}]{ #2 }\tikzexternalenable}
\newcommand{\onormaltodo}[2]{\tikzexternaldisable\todo[color=orange!30,caption={#1}]{ #2 }\tikzexternalenable}
}{%false
\newcommand{\toreader}[2]{\todo[inline,color=blue!30,caption={#1}]{ #2 }}
\newcommand{\critical}[2]{\todo[inline,color=red!30,caption={#1}]{ #2 }}
\newcommand{\related}[2]{\todo[inline,color=green!30,caption={#1}]{ #2 }}
\newcommand{\normaltodo}[2]{\todo[inline,color=orange!30,caption={#1}]{ #2 }}
\newcommand{\otoreader}[2]{\todo[color=blue!30,caption={#1}]{ #2 }}
\newcommand{\ocritical}[2]{\todo[color=red!30,caption={#1}]{ #2 }}
\newcommand{\orelated}[2]{\todo[color=green!30,caption={#1}]{ #2 }}
\newcommand{\onormaltodo}[2]{\todo[color=orange!30,caption={#1}]{ #2 }}
}

%----------------------------------------------------------------------------------------
%	FRONT PAGE
%----------------------------------------------------------------------------------------
\etitle{A Lua-LaTeX thesis template for IS/CS at U-Tokyo}
\jtitle{東大IS/CSのLua-LaTeXのテンプレート}

\eauthor{TODAI Taro}
\jauthor{東大太郎}

\esupervisor{HONGO Hanako}
\jsupervisor{本郷花子}
\supervisortitle{Professor} % Professor, etc.

\date{\today}%February XX, 20XX

%----------------------------------------------------------------------------------------
%----------------------------------------------------------------------------------------
%	COMMAND FORCED TO BE AT END OF THE PREAMBLE
%----------------------------------------------------------------------------------------
%----------------------------------------------------------------------------------------

% Reference packages (in this order, almost always after every other package)
\usepackage[colorlinks=false, pdfborder={0 0 0}]{hyperref}
\usepackage{cleveref} % Call \cref for lower-case reference and /Cref for upper-case
%\crefname{environment}{singular}{plural} to redefine the output of \cref for this environment
\crefname{prop}{proposition}{propositions}
\crefname{cor}{corollary}{corollaries}
\crefname{lem}{lemma}{lemmata}
\crefname{thm}{theorem}{theorems}
\crefname{defn}{definition}{definitions}
\crefname{conj}{conjecture}{conjectures}
\crefname{exmp}{example}{examples}
\crefname{rem}{remark}{remarks}
\crefname{note}{note}{notes}
\crefname{case}{case}{cases}

% Natbib bibliography package (after hyperref)
\usepackage[longnamesfirst,colon,square,sort]{natbib}
\renewcommand{\harvardurl}[1]{\textbf{URL:} \url{#1}} % To make _ compatible

% Glossary (after hyperref)
\usepackage[xindy,toc,acronym,nowarn]{glossaries}
\makeglossaries
\input{glossary/glossary}

\begin{document}
\begin{eabstract}\toreader{English Abastract}{The abstract is a summary of the whole thesis. It presents all the major elements of your work in a highly condensed form. It is also the first impression that your reader will get, so it is very important to ensure that is well written and provides only the important information in an informative, interesting and succinct manner.}
\ 
\normaltodo{Instructions about the English abstract}{A thesis abstract is longer than in articles (1-2 pages, approx. 175 to 450 words).}
\ 
\critical{Important in English abstract}{Key points:
\begin{itemize}
\item The most common error in abstracts is failure to present results, approximately the last half of the abstract should be dedicated to summarizing and interpreting your them.
\item Your abstract should not normally contain citations.
\item Although you may have written an abstract as a 'thinking tool' earlier in your writing process, you need to write the final version after you have completed the thesis so that you have a good understanding of the findings and can clearly explain your contribution to the field
\end{itemize}}\end{eabstract}
\begin{jabstract}\toreader{Japanese Abastract}{The English abstract will be succeded by a Japanese translation of it.}
\ 
\normaltodo{Instructions about the Japanese abstract}{The length will be slightly shorter than the English abstract since Japanese text are more dense.}
\ 
\critical{Important in Japanese abstract}{Remember to:
\begin{itemize}
\item If you are not a native speaker you might want to do some external translation/read-proofs.
\end{itemize}}\end{jabstract}

\maketitle

\begin{acknowledge}\toreader{Acknowledgements}{This pages is for giving recognition to all people who has contributed to some extension in helping you to complete the hard task of writting a thesis.}
\ 
\normaltodo{Instructions about the acknowledgements}{Keep it extra short (1-2 pages).}
\ 
\critical{Important in acknowledgements}{Remember to mention at least:
\begin{itemize}
\item Supervisor (and co-supervisor)
\item Thesis commitee members
\item Co-authors of your publications
\item Members (current or past) of the lab that help you during your thesis
\item Project and project members (if was done as part of a larger group work)
\item Companies and organizations (if was done in collaboration with any company, or provide you with any resource [data, software, ...])
\item Scholarship and other founding recived (if any)
\end{itemize}

And do not forget to use the proper titles (Prof., Asst. Prof., Assoc. Prof., Ph.D., Mr., ...) for each one.}\end{acknowledge}

\frontmatter %% 前付け
\tableofcontents % 目次

%\todototoc % To add todolist to toc
%\listoftodos % To list the todo notes
%-------------------
\mainmatter %% 本文 % Begin numeric (1,2,3...) page numbering

%\toreader{To the reader}{To make the redaction and the reading of this draft thesis easier, I will use different colored notes.
%\begin{itemize}
 %\item Blue indicates notes at the attention of the reader
 %\item Red indicates errors or critical parts that need to be fixed asap
 %\item Green indicates lack of references or matter that need to be checked from other works
 %\item Orange concerns normal (other) notes
%\end{itemize}
%If any reader has the need to correct directly the latex document, please be consistent with the notes colors. The macro to use them can be found in the header of the main file.
%}
%\critical{Example of critical ToDoNotes}{Example of critical ToDoNotes, normally in red.}
%\related{Example of related work ToDoNotes}{Example of related work ToDoNotes, normally in green.}
%\normaltodo{Example of normal ToDoNotes}{Example of normal ToDoNotes, normally in orange.}

%\normaltodo{List of Latex macros or environment to implement}{
%List of Latex macros or environment to implement:
%\begin{itemize}
 %\item Environment for Linear Program or alike
%\end{itemize}

%}

%----------------------------------------------------------------------------------------
%	CHAPTERS (by input)
%----------------------------------------------------------------------------------------
\chapter{Introduction}
\label{chapter:introduction}

\toreader{Introduction}{This chapter gives a general overview to the problem, definition, historical background, major achivements and relevance.}
\ 
\normaltodo{Instructions about the introduction}{The introduction must be short (5-10 pages).}
\ 
\critical{Important in introduction}{Remember:
\begin{itemize}
\item All technical details of the problem should be presented in the \cref{chapter:preliminaries}
\item Do several references to others researchers' work in the introduction  \cite{Komaba2011online, Hongo2004techreport, Kashiwa1988book}
\end{itemize}}


\chapter{Preliminaries}
\label{chapter:preliminaries}

\toreader{Preliminaries}{A.}
\ 
\normaltodo{Instructions about the preliminaries}{A.}
\ 
\critical{Important in the preliminaries}{Remember to:
\begin{itemize}
\item A
\end{itemize}}
\chapter{Contribution I}
\label{chapter:contribution_I}

\toreader{Contributions}{A.}
\ 
\normaltodo{Instructions about each contribution chapter}{Each of yours significant independent contributions should have its own chapter  (10-40 pages).}
\ 
\critical{Important in the contribution chapters}{Remember to:
\begin{itemize}
\item Each chapter should be self-contained (using only things of the \cref{chapter:preliminaries})
\item And reference the publications of your own where each idea was first published \cite{todai2014incollection, todai2012article, todai2013inproceedings}
\end{itemize}}


\chapter{Experimental results}
\label{chapter:experiments}

\toreader{Experimental results}{A.}
\ 
\normaltodo{Instructions about the experimental results}{A.}
\ 
\critical{Important in the experimental results}{Remember to:
\begin{itemize}
\item A
\end{itemize}}
\chapter{Conclusions}
\label{chapter:conclusions}

\toreader{Conclusions}{A.}
\ 
\normaltodo{Instructions about the conclusions}{A.}
\ 
\critical{Important in the conlcusions}{Remember to:
\begin{itemize}
\item A
\end{itemize}}

\backmatter

%----------------------------------------------------------------------------------------
%	BIBLIOGRAPHY
%----------------------------------------------------------------------------------------

\label{Bibliography}

\nocite{*} % To print all the references without citation

\bibliographystyle{unsrtnat}
%\bibliographystyle{plain} % 参考文献
\bibliography{bibliography/bibliography} %

%----------------------------------------------------------------------------------------
%	APPENDICES
%----------------------------------------------------------------------------------------
\ifbool{print_appendix}{%true
  \begin{appendices}
    \chapter[Source code]{Source code}
\label{appendix:code}

\toreader{Source code appendix}{A.}
\ 
\normaltodo{Instructions about the source code appendix}{A.}
\ 
\critical{Important in the source code appendix}{Remember to:
\begin{itemize}
\item A
\end{itemize}}

  \end{appendices}
}{}%false

%----------------------------------------------------------------------------------------
%	LISTS
%----------------------------------------------------------------------------------------
\listoffigures % 図目次
\listoftables % 表目次
%\lstlistoflistings % ソースコード目次

%----------------------------------------------------------------------------------------
%	GLOSSARY
%----------------------------------------------------------------------------------------
\printglossaries

%-------------------
\end{document}
